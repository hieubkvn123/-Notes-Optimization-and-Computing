\usepackage{tcolorbox}
\usepackage{amssymb}
\tcbuselibrary{theorems}
\usepackage{amsmath,amsthm}
\usepackage{thmtools}
\usepackage[nottoc]{tocbibind}
\usepackage[english]{babel}
\usepackage{xhfill}
\usepackage{changepage}
\usepackage{listings}
\usepackage{xcolor}
\usepackage{framed}
\usepackage{etoolbox}
\usepackage[
    style=authoryear,
    maxcitenames=1
]{biblatex} 
\usepackage{xcolor}
\usepackage{fancyhdr}
\usepackage{booktabs}
\usepackage{amsmath}
\usepackage{csquotes}
\usepackage{tikz}
\usepackage{eso-pic}
\usepackage{mathabx}
\usepackage{stackengine}
% ------------------------------ %
\usepackage{hyperref}
\usepackage{cleveref}

%% Set bib style
\addbibresource{main.bib} 

%% Setting up hyperref options %%
\hypersetup{
    colorlinks,
    linkcolor={red!50!black},
    citecolor={blue!50!black},
    urlcolor={blue!80!black}
}

% Composition operator
\DeclareFontFamily{U}{mathx}{\hyphenchar\font45}
\DeclareFontShape{U}{mathx}{m}{n}{
    <-6> mathx5
    <6-7> mathx6
    <7-8> mathx7
    <8-9> mathx8
    <9-10> mathx9
    <10-12> mathx10
    <12-> mathx12
}{}
\DeclareMathSymbol{\bigovoid}{\mathop}{mathx}{"EC}
\DeclareMathOperator*{\esssup}{ess\,sup}
\newcommand{\bigO}{\mathop{\stackinset{c}{}{c}{}{ \scalebox{1.1}{$\bigovoid$}}{ \scalebox{1.15}{$\bigovoid$}}}}



%% Theorem style %%
\newtheoremstyle{definition}
                {2pt} % top space
                {2pt} % bottom space
                {} % font
                {0pt} % indent
                {\bfseries} % theorem name font
                {. \hrulefill}
                {\newline}
                {\thmname{#1}\thmnumber{ #2}\textnormal{\thmnote{ (#3)}}}

\newenvironment{subproof}[1]
{
    \begin{adjustwidth}{0pt}{}
    \textbf{#1}
    \newline
}
{
    \end{adjustwidth}
}
\newtheoremstyle{proof}
                {0pt} % top space
                {15pt} % bottom space
                {} % font
                {0pt} % indent
                {\bfseries} % theorem name font
                {. \hrulefill}
                {\newline}
                {\thmname{#1}\thmnumber{ #2}\textnormal{\thmnote{ (#3)}}}

\newtcbtheorem[number within=section, list inside={proplist}]
    {proposition}% name
    {Proposition}% title
    {%
        colback=red!5,
        colframe=red!35!black,
        fonttitle=\bfseries,
    }% options
    {prop}% prefix

\newtcbtheorem[number within=section, list inside={theoremlist}]
    {theorem}% name
    {Theorem}% title
    {%
        colback=blue!5,
        colframe=blue!35!black,
        fonttitle=\bfseries,
    }% options
    {thm}% prefix

\newtcbtheorem[number within=section, list inside={lemlist}]
    {lemma}% name
    {Lemma}% title
    {%
        colback=yellow!5,
        colframe=yellow!35!black,
        fonttitle=\bfseries,
    }% options
    {lem} % prefix

\newtcbtheorem[number within=subsection]
    {exercise}% name
    {Exercise}% title
    {%
        colback=black!5,
        colframe=black!35!black,
        fonttitle=\bfseries,
    }% options
    {ex}% prefix

\newtcbtheorem[number within=section, list inside={corolist}]
    {corollary}% name
    {Corollary}% title
    {%
        colback=cyan!5,
        colframe=cyan!35!black,
        fonttitle=\bfseries,
    }% options
    {coro} % prefix

%% Remarks %%
\newtheorem{remark}{Remark}

%% Proofs %%
\theoremstyle{proof}
\newtheorem*{pf*}{Proof}
\newenvironment{proof*}{
\begin{pf*}
}{
\hfill $\square{}$.
\end{pf*}
}

%% Exercise solution %%
\theoremstyle{proof}
\newtheorem*{sol*}{Solution}
\newenvironment{solution*}{
\begin{sol*}
}{
\hfill $\square{}$.
\end{sol*}
}

%% Definitions %%
\theoremstyle{definition}
\declaretheorem[name=Definition, numberwithin=section]{dfn}
\addto\captionsenglish{ \renewcommand{\listtheoremname}{A \ List of Definitions} }
\newenvironment{definition}{\begin{dfn}}{\end{dfn}}

%% Indicator function %%
\DeclareMathAlphabet{\mathmybb}{U}{cmr}{m}{n}
\newcommand{\1}[1]{\mathmybb{\bf 1}_{\{#1\}}}

%% Brackets %%
\newcommand{\bigSquare}[1]{
\Big[ #1 \Big]
}
\newcommand{\biggSquare}[1]{
\Bigg[ #1 \Bigg]
}

\newcommand{\bigRound}[1]{
\Big( #1 \Big)
}
\newcommand{\biggRound}[1]{
\Bigg( #1 \Bigg)
}

\newcommand{\bigCurl}[1]{
\Big\{ #1 \Big\}
}
\newcommand{\biggCurl}[1]{
\Bigg\{ #1 \Bigg\}
}

\newcommand{\bigAbs}[1]{
\Big| #1 \Big|
}
\newcommand{\biggAbs}[1]{
\Bigg| #1 \Bigg|
}

%% Some shortcuts %%
% Real numbers
\newcommand{\R}{\mathbb{R}}
\newcommand{\E}{\mathbb{E}}
\newcommand{\Z}{\mathbb{Z}}

% Function class
\newcommand{\Hf}{\mathcal{H}}

% Feature set
\newcommand{\X}{\mathcal{X}}

% Label set
\newcommand{\Y}{\mathcal{Y}}

% Cover
\newcommand{\C}{\mathcal{C}}

% Rademacher complexity
\newcommand{\RC}{\mathfrak{R}}
\newcommand{\ERC}{\widehat{\mathfrak{R}}}

% Others
\newcommand{\F}{\mathcal{F}}
\newcommand{\G}{\mathcal{G}}
\newcommand{\vecbf}[1]{{\bf #1}}

% Inner product
\newcommand{\inner}[1]{\big<#1\big>}

% To denote layer number in neural networks
\newcommand{\A}[1]{A^{(#1)}}
\newcommand{\bA}[1]{{\bar A}^{(#1)}}
\newcommand{\M}[1]{M^{(#1)}}

% Outer measure
\newcommand{\om}{\mu^*}


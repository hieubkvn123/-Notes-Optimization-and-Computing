\newcommand{\cvxopt}{{\bf CVXOPT}}
\section{Introduction}
\begin{definition}[Optimization Problem]
	Generally, an optimization problem is defined as follows:
	\begin{equation}\begin{aligned}
		\textup{minimize}: 		& \ f_0(x) \\
		\textup{subject to}: 	& \ f_i(x) \le 0, \quad i = 1, \dots, m \\
								& \ h_i(x) = 0, \quad i = 1, \dots, p.	
	\end{aligned}\end{equation} 
	\noindent Where we have:
	\begin{enumerate}
		\item $x\in\R^n$ is the optimization variable.
		\item $f_0:\R^n\to\R$ is the opjective (cost function).
		\item $f_i:\R^n\to\R$ are inequality constraints.
		\item $h_i:\R^n\to\R$ are equality constraints.
	\end{enumerate} 
\end{definition} 

\begin{definition}[Convex Optimization Problem]
	\noindent An optimization problem is a \textbf{convex optimization problem} if:
	\begin{enumerate}
		\item $f_0, f_1, \dots, f_m$ are convex.
		\item Equality constraints are affine.
	\end{enumerate} 
\end{definition} 

\noindent The reason why we need convex optimization problems are:
\begin{enumerate}
	\item Convex optimization problems can be solved optimally (no local minima).
	\item Time required to solve convex optimization problems is polynomial (in terms of number of variables and constraints).
\end{enumerate} 

\subsection{Convex Sets}
\begin{definition}[Lines]
	Let $x_1, x_2\in\R^n$. A line passing through $x_1, x_2$ is defined as:
	\begin{equation}
		L(x_1, x_2) = \bigCurl{
			x\in\R^n: x = \theta x_1 + (1-\theta)x_2, \theta\in\R
		}.
	\end{equation}

	\noindent When $\theta\in(0,1)$, we restrict the line to the points between $x_1$ and $x_2$ (exclusive). 
\end{definition} 

\begin{definition}[Affine Sets]
	An affine is a set that contains the line segment between any two distinct points in it. For example,
	\begin{enumerate}
		\item An empty set is affine because there is no point.
		\item A singleton is affine because there is only one point.
		\item A line (extends indefinitely) is affine.
		\item Any vector space is affine.
		\item Linear subspaces of a vector space is affine.	
	\end{enumerate} 

	\noindent In other words, an affine set contains its elements' \textbf{affine combinations}: If $x_1, \dots, x_k$ belongs to an affine set $A$, then it contains the affine combination
	\begin{equation}
		\sum_{i=1}^k\theta_i x_i\in A, \quad \theta_i\in\R, \sum_{i=1}^k\theta_i = 1.
	\end{equation} 
\end{definition} 

\begin{definition}[Convex Sets]
	A convex set contains its elements' \textbf{convex combinations}: If $x_1, \dots, x_k$ belongs to an affine set $A$, then it contains the convex combination
	\begin{equation}
		\sum_{i=1}^k\theta_i x_i\in A, \quad \theta_i\in [0, 1], \sum_{i=1}^k\theta_i = 1.	
	\end{equation} 
\end{definition} 



